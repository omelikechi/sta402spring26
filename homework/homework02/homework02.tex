\documentclass[11pt,oneside,reqno]{amsart}

% packages
\usepackage{algorithm, algpseudocode, amsfonts, amssymb, amsthm, amsmath, array, authblk, bm, braket, caption, changepage, comment, dsfont, enumitem, etoolbox, fullpage, graphicx, mathrsfs, mathtools, microtype, multicol, pifont, pythonhighlight, relsize, subfig, subfloat, textcomp, tikz, tikz-cd, url, xpatch}

\usepackage{hyperref}
\hypersetup{
	colorlinks=true,
	linkcolor=blue,
	citecolor=black,
	urlcolor=blue}
\usepackage[noabbrev,capitalise]{cleveref}
\creflabelformat{equation}{#2\textup{#1}#3}
\newcommand{\creflastconjunction}{, and\nobreakspace}
\numberwithin{equation}{section}

%\usepackage[round]{natbib}

% format
\setlength\parindent{0pt}
\setlength{\parskip}{\baselineskip}
\setlist[enumerate]{itemsep=1ex, topsep=-1ex}
\setlist[itemize]{itemsep=0pt, topsep=0pt, leftmargin=20pt}

% footnotes
%\usepackage[hang,flushmargin]{footmisc}
\usepackage[foot]{amsaddr}

% font
%\usepackage[T1]{fontenc}
%\usepackage{newpxtext} \usepackage[euler-digits]{eulervm}
%\usepackage[euler-digits]{eulervm}
%\usepackage{mathpazo}

% sections
\usepackage{titlesec}

\titleformat{\section}
	{\titlerule\vspace{-2ex}}
	{\thesection.}{1ex}{\centering\scshape}
	[\vspace{.5ex}\titlerule]
	
\titleformat{\subsection}[runin]
  {\normalfont\normalsize\bfseries}{\thesubsection.}{1ex}{}	
\titlespacing*{\subsection}{0pt}{0.0\baselineskip}{0.5ex}

% theorems
\newtheorem{theorem}{Theorem}[section]

\newtheoremstyle{style}
  {\baselineskip} % Space above
  {0em} % Space below
  {\itshape} % Body font
  {} % Indent amount
  {\bfseries} % Theorem head font
  {.} % Punctuation after theorem head
  {.5em} % Space after theorem head
  {} % Theorem head spec (can be left empty, meaning `normal')
\theoremstyle{style}

\newtheorem*{theorem*}{Theorem}

\numberwithin{equation}{section}
\newtheorem{assumption}{Assumption}
\newtheorem{condition}{Condition}
\newtheorem{cor}[theorem]{Corollary}
\newtheorem{definition}[theorem]{Definition}
\newtheorem*{definition*}{Definition}
\newtheorem{example}[theorem]{Example}
\newtheorem{fig}{Figure}[section]
\newtheorem{lemma}[theorem]{Lemma}
\newtheorem*{lemma*}{Lemma}
\newtheorem{prop}[theorem]{Proposition}
\newtheorem*{prop*}{Proposition}
\newtheorem{remark}[theorem]{Remark}

\crefname{condition}{Condition}{Conditions}

% proofs
\newcommand{\QED}{\tag*{\qed}}

% algorithms
\algrenewcommand\algorithmicrequire{\textbf{Input:}}
\algrenewcommand\algorithmicensure{\textbf{Output:}}

% lists
\setlist[itemize]{itemsep=0.5em, left=0.5em}

% notation
% \DeclareMathOperator{shorthand}{expression}
\DeclareMathOperator{\1}{\mathds{1}}
\DeclareMathOperator*{\argmax}{arg\,max}
\DeclareMathOperator*{\argmin}{arg\,min}
\newcommand{\iid}{\overset{\mathrm{iid}}{\sim}}
\newcommand{\var}{\text{Var}}

\newcommand{\blue}[1]{\textcolor{blue}{#1}}
\newcommand{\green}[1]{\textcolor{green}{#1}}
\newcommand{\red}[1]{\textcolor{red}{#1}}

%-------------------------------------------------------------------

\title{{\large {\Large S}TA402L: {\Large H}omework 2}}
\author{{\large Due: 11:59 pm on XXX}}

\begin{document}

\frenchspacing

\maketitle

\textbf{Instructions.} Solutions must be submitted to Gradescope as a single PDF. Programming exercises must be completed in R, should be clearly presented, and include all R code. Lab questions are restated here for convenience, but you should refer to the lab itself for details.

\textbf{Total points.} Book exercises: XXX; Lab exercises XXX; Overall: XXX.

%%%%%%%%%%%%%%%%%%%%%%%%%%%%%%%%%%%%%%%%%%%%%%%%%%%%%%%%%%%%%%%%

\section*{Book exercises}

%%%%%%%%%%%%%%%%%%%%%%%%%%%%%%%%%%%%%%%%%%%%%%%%%%%%%%%%%%%%%%%%

\begin{enumerate}[label=B\arabic*., leftmargin=2em, itemsep=2em]

\item (3 points) Show that the posterior variance of the beta--binomial model can be written as
\begin{align*}
\var(\theta \mid y) &= \frac{\mathbb{E}(\theta \mid y)\,\mathbb{E}(1-\theta \mid y)}{a+b+n+1}\,.
\end{align*}


\item Hoff 3.1.

\vspace*{1em}

\begin{enumerate}[label=(\alph*), leftmargin=1em, itemsep=1em]
\item (1 point)
\item (1 point)
\item (1 point)
\item (1 point)
\item (2 points)
\end{enumerate}

\end{enumerate}

%%%%%%%%%%%%%%%%%%%%%%%%%%%%%%%%%%%%%%%%%%%%%%%%%%%%%%%%%%%%%%%%

\section*{Lab exercises}

%%%%%%%%%%%%%%%%%%%%%%%%%%%%%%%%%%%%%%%%%%%%%%%%%%%%%%%%%%%%%%%%

\begin{enumerate}[label=L\arabic*., leftmargin=2em, itemsep=2em]

\item (1 point) Plot a histogram of $\theta$ from the \texttt{rstan} object called \texttt{pool\_output}. Describe the distribution.

\item Visualize the posterior distributions of the $\theta_i$ with boxplots. In the plot, there should be one box and whiskers object for each $\theta_i$.

\item Take a few minutes to look at the contents of the two files \texttt{lab-02-pool.stan} and \texttt{lab-02-nopool.stan}. How are they different?
\end{enumerate}

\end{document}


