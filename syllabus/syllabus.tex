\documentclass[11pt,oneside,reqno]{amsart}

% packages
\usepackage{algorithm, algpseudocode, amsfonts, amssymb, amsthm, amsmath, array, authblk, bm, braket, caption, changepage, comment, dsfont, enumitem, etoolbox, fullpage, graphicx, mathrsfs, mathtools, microtype, multicol, pifont, pythonhighlight, relsize, subfig, subfloat, textcomp, tikz, tikz-cd, url, xpatch}

\usepackage{hyperref}
\hypersetup{
	colorlinks=true,
	linkcolor=blue,
	citecolor=black,
	urlcolor=blue}
\usepackage[noabbrev,capitalise]{cleveref}
\creflabelformat{equation}{#2\textup{#1}#3}
\newcommand{\creflastconjunction}{, and\nobreakspace}
\numberwithin{equation}{section}

%\usepackage[round]{natbib}

% format
\setlength\parindent{0pt}
\setlength{\parskip}{\baselineskip}
\setlist[enumerate]{itemsep=1ex, topsep=-1ex}
\setlist[itemize]{itemsep=0pt, topsep=0pt, leftmargin=20pt}

% footnotes
%\usepackage[hang,flushmargin]{footmisc}
\usepackage[foot]{amsaddr}

% font
%\usepackage[T1]{fontenc}
%\usepackage{newpxtext} \usepackage[euler-digits]{eulervm}
%\usepackage[euler-digits]{eulervm}
%\usepackage{mathpazo}

% sections
\usepackage{titlesec}

\titleformat{\section}
	{\titlerule\vspace{-2ex}}
	{\thesection.}{1ex}{\centering\scshape}
	[\vspace{.5ex}\titlerule]
	
\titleformat{\subsection}[runin]
  {\normalfont\normalsize\bfseries}{\thesubsection.}{1ex}{}	
\titlespacing*{\subsection}{0pt}{0.0\baselineskip}{0.5ex}

% theorems
\newtheorem{theorem}{Theorem}[section]

\newtheoremstyle{style}
  {\baselineskip} % Space above
  {0em} % Space below
  {\itshape} % Body font
  {} % Indent amount
  {\bfseries} % Theorem head font
  {.} % Punctuation after theorem head
  {.5em} % Space after theorem head
  {} % Theorem head spec (can be left empty, meaning `normal')
\theoremstyle{style}

\newtheorem*{theorem*}{Theorem}

\numberwithin{equation}{section}
\newtheorem{assumption}{Assumption}
\newtheorem{condition}{Condition}
\newtheorem{cor}[theorem]{Corollary}
\newtheorem{definition}[theorem]{Definition}
\newtheorem*{definition*}{Definition}
\newtheorem{example}[theorem]{Example}
\newtheorem{fig}{Figure}[section]
\newtheorem{lemma}[theorem]{Lemma}
\newtheorem*{lemma*}{Lemma}
\newtheorem{prop}[theorem]{Proposition}
\newtheorem*{prop*}{Proposition}
\newtheorem{remark}[theorem]{Remark}

\crefname{condition}{Condition}{Conditions}

% proofs
\newcommand{\QED}{\tag*{\qed}}

% algorithms
\algrenewcommand\algorithmicrequire{\textbf{Input:}}
\algrenewcommand\algorithmicensure{\textbf{Output:}}

% lists
\setlist[itemize]{itemsep=0.5em, left=0.5em}

% notation
% \DeclareMathOperator{shorthand}{expression}
\DeclareMathOperator{\1}{\mathds{1}}
\DeclareMathOperator*{\argmax}{arg\,max}
\DeclareMathOperator*{\argmin}{arg\,min}
\newcommand{\iid}{\overset{\mathrm{iid}}{\sim}}

\newcommand{\blue}[1]{\textcolor{blue}{#1}}
\newcommand{\green}[1]{\textcolor{green}{#1}}
\newcommand{\red}[1]{\textcolor{red}{#1}}

%-------------------------------------------------------------------

\title{{\Large {\LARGE S}TA402L: {\LARGE B}ayesian Modeling}}
\author{{\large D}uke {\large U}niversity, {\large S}pring {\large 2026}}

\begin{document}

\frenchspacing

\maketitle

%{\large\textbf{Schedule}}
\begin{table}[ht]
\raggedright
\renewcommand{\arraystretch}{1.5}      % Controls vertical row spacing
\setlength{\arrayrulewidth}{1pt}     % Controls line thickness

\begin{tabular}{p{2.5cm} p{2cm} p{3.5cm} p{3.5cm}}
\hline
 & \textbf{Day} & \textbf{Time} & \textbf{Location} \\
\hline
Lectures & Wed/Fri & 11:45am--1:00pm & Old Chem 116 \\
Lab 1 & Thu & 1:25pm--2:40pm & Old Chem 101 \\
Lab 2 & Thu & 3:05pm--4:20pm & Old Chem 101 \\
\hline
\end{tabular}
\end{table}

\vspace*{1em}

%{\large\textbf{People}}
\begin{table}[ht]
\raggedright
\renewcommand{\arraystretch}{1.5}      % Controls vertical row spacing
\setlength{\arrayrulewidth}{1pt}     % Controls line thickness

\begin{tabular}{p{3cm} p{4cm} p{4.5cm} p{3cm}}
\hline
 & \textbf{Contact} & \textbf{Office Hours} & \textbf{Location} \\
\hline
Omar Melikechi & oem2@duke.edu & W/F: 10:00am--11:00am & Old Chem 122 \\
Yihao Gu & yihao.gu@duke.edu & TBD & TBD \\
Sonya Eason & sonya.eason@duke.edu & TBD & TBD  \\
\hline
\end{tabular}
\end{table}

\vspace*{1em}

\textbf{Course website.} \href{https://omelikechi.github.io/sta402spring26/}{https://omelikechi.github.io/sta402spring26/}

\textbf{Textbook.} ``\href{https://pdhoff.github.io/book/}{\textit{A first course in Bayesian statistical methods}}" by Peter Hoff. Duke students can download an electronic version for free from Duke Library.

\textbf{Additional reading (optional).} ``\href{https://sites.stat.columbia.edu/gelman/book/}{\textit{Bayesian data analysis}}'' by Andrew Gelman, John Carlin, Hal Stern, David Dunson, Aki Vehtari, and Donald Rubin.

\textbf{Homework.} Homework assignments are posted on the course website. Homework solutions must be uploaded to \href{https://www.gradescope.com/courses/1205917}{the course Gradescope page} as a single PDF file. $25\%$ of each assignment's total score will be deducted per day after that assignment's due date. Regrade requests for a particular assignment must be made within one week of receiving your grade on that assignment.

\textbf{Lab.} Lab exercises are to be completed and turned in as part of homework.

\textbf{Exams.} There will be two midterms and one final. The final exam is on Friday, May 1, 2026 from 2:00pm to 5:00pm. There will be no make-up exams.

\textbf{Course grade.} Homework (25\%), Midterm 1 (25\%), Midterm 2 (25\%), Final (25\%).

\end{document}


